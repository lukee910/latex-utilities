\RequirePackage{currfile}
\let\lutilpath\currfiledir
\newcommand{\templates}{\lutilpath templates}
% === Document class
\documentclass[a4paper, twocolumn]{article}

% === Packages
\usepackage[ngerman]{babel}
\usepackage[margin=2.5cm]{geometry}

% === Utilites from this repo
% Sequences
\newcommand{\seq}[3][1]{(#2_{#3})_{#3 \geq #1}}
\newcommand{\seqin}[3][\nat]{(#2_{#3})_{#3 \in #1}}

% Series
\newcommand{\series}[3][1]{\sum_{#3=#1}^{\infty} #2_#3}
\newcommand{\absseries}[3][1]{\sum_{#3=#1}^{\infty} \left| #2_#3 \right|}

\newcommand{\dx}{\text{ d}x}
\newcommand{\dy}{\text{ d}y}
% Number sets
\newcommand{\nat}{\mathbb{N}}
\newcommand{\whole}{\mathbb{Z}}
\newcommand{\rat}{\mathbb{Z}}
\newcommand{\real}{\mathbb{R}}
\newcommand{\comp}{\mathbb{C}}

% Limits
\newcommand{\limtoinf}[1][x]{\limits_{#1 \to \infty}}
\newcommand{\limto}[2][x]{\limits_{#1 \to #2}}
\newcommand{\limtoneq}[2][x]{\limits_{#1 \to #2,\ #1 \neq #2}}

% O-Notation
\DeclareRobustCommand{\bigO}{
  \text{\usefont{OMS}{cmsy}{m}{n}O}
}


% === Commands, definitions, etc.

% === Template styling
% Theorems
\RequirePackage{\currfiledir../../styling/theorems/basic}



\title{Basic Summary Example}
\author{Lukas Dörig}

\begin{document}

\maketitle

\section{Boxes}

\subsection{Basic Boxes}

This is a sentence.

\begin{mainbox}{A main box title}
  Some main box content
\end{mainbox}

\begin{subbox}{A main sub title}
  Some sub box content
\end{subbox}

\subsection{Theorems etc.}

\begin{axiom}{Existence of boxes}{box}
  These boxes exist.
\end{axiom}

\begin{definition}{Box}{box}
  A box is a square with content.
\end{definition}

\begin{corollary}{Maths}{maths}
  Let $a \in \real,\ b \in \real$.

  \[
    a \geq b \Rightarrow a - b \geq 0
  \]
\end{corollary}

\setnext{lulemma}{6}
\begin{lemma}{Numbering lemma}{num}
  You can set the counter, so that the lemma will have a certain number.
\end{lemma}

\begin{satz}{Pythagoras}{triangles}
  A quadrat plus bee quadrat = sea quadrat.
\end{satz}

\begin{theorem}{Misnomer corollary}{misnomer}
  Theorems are never theorems and lemmas are never lemmas.
\end{theorem}

\subsection{Counting}

\begin{axiom}{Counting}{count}
  Latex can count from 1.1 to 1.2.
\end{axiom}

\section{References}

\begin{corollary}{References}{ref}
  You can reference stuff. E.g. definition \ref{def:box} with the title \nameref{def:box}.
\end{corollary}

\end{document}
