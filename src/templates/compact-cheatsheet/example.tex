\def\lutilpath{../..}

% === Document class
\documentclass[a4paper, twocolumn]{article}

% === Packages
\usepackage[ngerman]{babel}
\usepackage[margin=2.5cm]{geometry}

% === Utilites from this repo
% Sequences
\newcommand{\seq}[3][1]{(#2_{#3})_{#3 \geq #1}}
\newcommand{\seqin}[3][\nat]{(#2_{#3})_{#3 \in #1}}

% Series
\newcommand{\series}[3][1]{\sum_{#3=#1}^{\infty} #2_#3}
\newcommand{\absseries}[3][1]{\sum_{#3=#1}^{\infty} \left| #2_#3 \right|}

\newcommand{\dx}{\text{ d}x}
\newcommand{\dy}{\text{ d}y}
% Number sets
\newcommand{\nat}{\mathbb{N}}
\newcommand{\whole}{\mathbb{Z}}
\newcommand{\rat}{\mathbb{Z}}
\newcommand{\real}{\mathbb{R}}
\newcommand{\comp}{\mathbb{C}}

% Limits
\newcommand{\limtoinf}[1][x]{\limits_{#1 \to \infty}}
\newcommand{\limto}[2][x]{\limits_{#1 \to #2}}
\newcommand{\limtoneq}[2][x]{\limits_{#1 \to #2,\ #1 \neq #2}}

% O-Notation
\DeclareRobustCommand{\bigO}{
  \text{\usefont{OMS}{cmsy}{m}{n}O}
}


% === Commands, definitions, etc.

% === Template styling
% Theorems
\RequirePackage{\currfiledir../../styling/theorems/basic}



\title{Compact Cheatsheet Example}
\author{Lukas Dörig}

\begin{document}

% Needs to be input'ed inside \begin{document}
% Note: This only works if it is included AFTER \begin{document}. Therefore, it
% can't be used like a normal styling file.

\setlength{\abovedisplayskip}{3pt}
\setlength{\belowdisplayskip}{3pt}
\setlength{\abovedisplayshortskip}{3pt}
\setlength{\belowdisplayshortskip}{3pt}


\begin{multicols*}{3}

  \maketitle

  \part{Formulas}

  \h{Boxes}

  \hh{Basic Boxes}

  This is a sentence.

  \begin{mainbox}{A main box title}
    Some main box content
  \end{mainbox}

  \begin{subbox}{A main sub title}
    Some sub box content
  \end{subbox}

  \hh{Theorems etc.}

  \begin{axiom}{Existence of boxes}{box}
    These boxes exist.
  \end{axiom}

  \begin{remark}{Reference ids}{box}
    This is to demonstrate that reference IDs can be the same for different types.
  \end{remark}

  \begin{definition}{Spaced equation}{spaced-equation}
    This is a well spaced multiline equation. Compare this to this text.
    \begin{align*}
      \prob{X = x} & := 0.5 \cdot \text{Coinflip} \\
      \expect{X} & = 259.58
    \end{align*}
    There's also text after it.
  \end{definition}

  \begin{corollary}{Maths}{maths}
    Let $a \in \real,\ b \in \real$.
    \[
      a \geq b \Rightarrow a - b \geq 0
    \]
    Here's some text after the equation.
  \end{corollary}

  \setnext{lulemma}{6}
  \begin{lemma}{Numbering lemma}{num}
    You can set the counter, so that the lemma will have a certain number.
  \end{lemma}

  \begin{satz}{Pythagoras}{triangles}
    A quadrat plus bee quadrat = sea quadrat.
  \end{satz}

  \begin{theorem}{Misnomer corollary}{misnomer}
    Theorems are never theorems and lemmas are never lemmas.
  \end{theorem}

  \hh{Counting}

  \begin{axiom}{Counting}{count}
    Latex can count from 1.1 to 1.2.
  \end{axiom}

  \h{References}

  \begin{corollary}{References}{ref}
    You can reference stuff. E.g. remark \ref{remark:box} with the title \nameref{remark:box}. \\
    Different types have different ids. So you could also reference axiom \ref{axiom:box} \nameref{axiom:box}.
  \end{corollary}

\end{multicols*}

\end{document}
