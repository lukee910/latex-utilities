\def\lutilpath{../..}

% === Document class
\documentclass[10pt,landscape,a4paper]{article}

% === Packages
\usepackage[ngerman]{babel}
\usepackage[margin=0.5cm]{geometry}

% Multi col layout
\usepackage{multicol}

% === Utilites from this repo
\input{\lutilpath/packages/all}
\input{\lutilpath/maths/all}
\input{\lutilpath/styling/all}
\input{\lutilpath/formatting/all}

\input{\lutilpath/packages/all}
\input{\lutilpath/maths/all}
\input{\lutilpath/styling/all}
\input{\lutilpath/formatting/all}

\input{\lutilpath/packages/all}
\input{\lutilpath/maths/all}
\input{\lutilpath/styling/all}
\input{\lutilpath/formatting/all}

\input{\lutilpath/packages/all}
\input{\lutilpath/maths/all}
\input{\lutilpath/styling/all}
\input{\lutilpath/formatting/all}



% === Commands, definitions, etc.

% Textsize
% \footnotesize

% === Template styling
% Theorems
% Sources
% xyquadrat: https://xyquadrat.ch/2022/04/04/latex-boxes/
% Azurios: https://azurios.gitlab.io/posts/boxes/

% === General

\tcbset {
  base/.style={
      arc=0mm,
      bottomtitle=0.5mm,
      boxrule=0mm,
      colbacktitle=black!10!white,
      coltitle=black,
      fonttitle=\bfseries,
      left=2.5mm,
      leftrule=1mm,
      right=3.5mm,
      title={#1},
      toptitle=0.75mm
    }
}

% Define colors, overridable

\ifx\lutilcolbasic\undefined
  \def\lutilcolbasic{black!30!white}
\fi

% === Basic Boxes

\definecolor{brandblue}{rgb}{0.34, 0.7, 1}

\newtcolorbox{mainbox}[1]{
  colframe=brandblue,
  base={#1}
}

\newtcolorbox{subbox}[1]{
  colframe=\lutilcolbasic,
  base={#1}
}

% === Theorems etc.

% Axiom

\newcounter{lutilaxiom}

\newtcbtheorem[use counter=lutilaxiom, number within=section]{axiom}{Axiom}{
  base={#1},
  colframe=\lutilcolbasic,
}{ax}

% Definition

\newcounter{lutildef}

\newtcbtheorem[use counter=lutildef, number within=section]{definition}{Definition}{
  base={#1},
  colframe=\lutilcolbasic,
}{def}




\title{Compact Cheatsheet Example}
\author{Lukas Dörig}

\begin{document}

% Needs to be input'ed inside \begin{document}
% Note: This only works if it is included AFTER \begin{document}. Therefore, it
% can't be used like a normal styling file.

\setlength{\abovedisplayskip}{3pt}
\setlength{\belowdisplayskip}{3pt}
\setlength{\abovedisplayshortskip}{3pt}
\setlength{\belowdisplayshortskip}{3pt}


\begin{multicols*}{3}

  \maketitle

  \part{Formulas}

  \h{Boxes}

  \hh{Basic Boxes}

  This is a sentence.

  \begin{mainbox}{A main box title}
    Some main box content
  \end{mainbox}

  \begin{subbox}{A main sub title}
    Some sub box content
  \end{subbox}

  \hh{Theorems etc.}

  \begin{axiom}{Existence of boxes}{box}
    These boxes exist.
  \end{axiom}

  \begin{definition}{Spaced equation}{spaced-equation}
    This is a well spaced multiline equation. Compare this to this text.
    \begin{align*}
      \prob{X = x} & := 0.5 \cdot \text{Coinflip} \\
      \expect{X} & = 259.58
    \end{align*}
    There's also text after it.
  \end{definition}

  \begin{corollary}{Maths}{maths}
    Let $a \in \real,\ b \in \real$.
    \[
      a \geq b \Rightarrow a - b \geq 0
    \]
    Here's some text after the equation.
  \end{corollary}

  \setnext{lulemma}{6}
  \begin{lemma}{Numbering lemma}{num}
    You can set the counter, so that the lemma will have a certain number.
  \end{lemma}

  \begin{satz}{Pythagoras}{triangles}
    A quadrat plus bee quadrat = sea quadrat.
  \end{satz}

  \begin{theorem}{Misnomer corollary}{misnomer}
    Theorems are never theorems and lemmas are never lemmas.
  \end{theorem}

  \hh{Counting}

  \begin{axiom}{Counting}{count}
    Latex can count from 1.1 to 1.2.
  \end{axiom}

  \h{References}

  \begin{corollary}{References}{ref}
    You can reference stuff. E.g. definition \ref{def:box} with the title \nameref{def:box}.
  \end{corollary}

\end{multicols*}

\end{document}
